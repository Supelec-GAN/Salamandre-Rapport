% \begin{figure}[htb]
% \begin{center}
% %optional pour enlever un peu d'espace blanc de votre silhouette
% %\vspace{-.3cm}
%  %\includegraphics[keepaspectratio,width=0.5\textwidth]{fig/nomde}
% % analoog
% %\vspace{-0.6cm}
%  % \caption{ici un logo}
%  % \label{fig:ruglogo}
% %analoog
% %\vspace{-.6cm}
% \end{center}
% \end{figure}

% Utilisation du template
% \cite(reference à citer)
% \ref(figure à référencer)

%\bibliography{reference}

\chapter{Generative Adversarial Networks}

\paragraph{}
La première partie de cette étude nous a permis de maîtriser l'utilisation de réseaux en perceptron et de structurer une architecture logicielle efficace et souple pour l'étude des GAN. \\
Après avoir obtenu des résultats satisfaisants sur la reconnaissance de motif sur la base MNIST, nous étudions la génération de données à l'aide de réseaux de neurones en nous basant sur le concept de GAN, introduit par I. Goodfellow en 2016 \cite{nips-2014}
\section{Principe}
\paragraph{}
Le but du GAN est de générer des images que l’on ne pourra distinguer de « vraies » images, prises avec un appareil photo. Un GAN est constitué de deux réseaux de neurones : le Générateur (G) et le Discriminateur (D). Le Générateur a pour but de créer les images et le Discriminateur de déterminer si les images qu’on lui donne sont de « vraies » images ou ont été créées par le Générateur. Ces deux réseaux sont mis en compétition : le Générateur a pour but de tromper le Discriminateur tandis que le Discriminateur doit discerner les « fausses » images.\\
L’apprentissage du Discriminateur se fait à la fois sur des images générées par le Générateur et de « vraies » images, issues d’une banque d’images.\\
L’apprentissage du Générateur dépend aussi du Discriminateur : lorsqu’il génère une image, on la donne au Discriminateur pour voir si le Générateur a réussi à le tromper. L’apprentissage du Générateur se fait sur la réponse du Discriminateur.

\paragraph{}
De façon plus formelle, on travaille avec 3 distributions : $p_x$, la distribution idéale des vraies images, $p_{data}$, l’échantillon des vraies images et $p_{model}$ la distribution réalisée par les images issues du Générateur. Le but de l’apprentissage est de rapprocher $p_{model}$ de $p_x$. Comme $p_x$ nous est inconnu, on va plutôt s'approcher de $p_{data}$.

\paragraph{}
Notre algorithme d’apprentissage repose une fois encore sur la descente de gradient et la rétro-propagation des erreurs. Cependant, les fonctions d’erreur sont différentes. En effet, nous ne connaissons pas la distribution exacte et ne pouvons donc pas minimiser une « distance minimale » entre le résultat obtenu et le résultat désiré. 

\section{Apprentissage}
\paragraph{}
L’apprentissage se déroule ainsi :\\
Nous fournissons au Discriminateur des images x (de la BDD MNIST par exemple) et lui demandons de nous renvoyer un réel entre 0 et 1, qui représente son degré de confiance sur le fait que l’image fournie ai été tirée d’une banque de donnée authentique (D(x) = 1) ou du Générateur (D(x) =0).\\
Le Générateur, quant à lui, devra générer une image à partir d’un vecteur de bruit z. Cette image est ensuite jugée par le Discriminateur : D(G(z)) = 1 si le Générateur a dupé le Discriminateur et 0 sinon. Puis, on effectue une rétropropagation à travers les deux réseaux pour les faire apprendre.

\paragraph{}
Les réseaux utilisés sont ceux que nous avons utilisé auparavant : des perceptrons multi-couches. Le Discriminateur, en particulier, reprend la même structure qu’un perceptron utilisé pour reconnaître des chiffres manuscrits.

\section{Résultats}


\section{Améliorations}


