%!TEX root = main.tex
% \begin{figure}[htb]
% \begin{center}
% %optional pour enlever un peu d'espace blanc de votre silhouette
% %\vspace{-.3cm}
%  %\includegraphics[keepaspectratio,width=0.5\textwidth]{fig/nomde}
% % analoog
% %\vspace{-0.6cm}
%  % \caption{ici un logo}
%  % \label{fig:ruglogo}
% %analoog
% %\vspace{-.6cm}
% \end{center}
% \end{figure}

% Utilisation du template
% \cite(reference à citer)
% \ref(figure à référencer)

\chapter{Axes de Recherches : WGAN}

\paragraph*{Introduction : } Avec les différents résultats obtenus par nos premiers GAN, nous avons pu tirer, entre autres, deux conclusions importantes. Le GAN manque cruellement de stabilité, par exemple un petit changement de paramètre l'empêche de converger correctement), et de métriques pertinentes, c'est à dire que les scores des générateurs et des discriminateurs n'ont pas d'interprétations en termes de progrès de la qualité d'image perçus.\\
Les chercheurs se sont beaucoup attardés depuis 2016 sur la première question, en comparant par exemple les différents optimiseurs possible \cite{optimiser}, le deuxième point est moins souvent abordés. \\
L'article de 2017  Wassert\cite{wgan} propose une méthode qui, en s'éloignant légèrement de la philosophie original du papier \cite{Goodfellow-et-al-2016}, tente d'apporter une réponse à ces deux questions, en particulier une métrique pertinente.

\section{Problématique de la descente de gradient simultanée}

\section{L'approche Wasserstein GAN}

\section{Mise en œuvre}

\section{Réflexion sur l'approche}