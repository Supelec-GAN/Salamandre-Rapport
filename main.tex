%!TEX root = main.tex
%  THESISBOEK
%
%  Ce fichier fournit des définitions générales (mise en page) et regroupe les
%  séparer les fichiers LaTeX en un ensemble.
%
%  @author Erwin Six, David De Reu, Brecht Vermeulen
%

\documentclass[11pt,a4paper,oneside,notitlepage]{book}
\usepackage[french]{babel}
\usepackage[utf8]{inputenc}
\usepackage[T1]{fontenc}

% ajustement des marges 
% (opmerking: moet *voor* inclusie van fancyhdr package komen)
\setlength{\hoffset}{-1in}
\setlength{\voffset}{-1in}
\setlength{\topmargin}{2cm}
\setlength{\headheight}{0.5cm}
\setlength{\headsep}{1cm}
\setlength{\oddsidemargin}{2.5cm}
\setlength{\evensidemargin}{2.5cm}
\setlength{\textwidth}{16cm}
\setlength{\textheight}{23.3cm}
\setlength{\footskip}{1.5cm}

\usepackage{amsmath}
\usepackage{amssymb}

\usepackage{bbold}
\usepackage{fancyhdr}
\usepackage{graphicx}
% \usepackage[colorlinks]{hyperref}

\pagestyle{fancy}

\renewcommand{\chaptermark}[1]{\markright{\MakeLowercase{#1}}}
\renewcommand{\sectionmark}[1]{\markright{\thesection~#1}}

\newcommand{\headerfmt}[1]{\textsl{\textsf{#1}}}
\newcommand{\headerfmtpage}[1]{\textsf{#1}}

%Pour faire des jolis R
\def\R{\textrm{I\kern-0.21emR}}

\fancyhf{}
\fancyhead[L,R]{\headerfmtpage{\thepage}}
\fancyhead[L]{\headerfmt{\rightmark}}
\fancyhead[R]{\headerfmt{\leftmark}}
\renewcommand{\headrulewidth}{0.5pt}
\renewcommand{\footrulewidth}{0pt}


\fancypagestyle{plain}{ % première page d'un chapitre
  \fancyhf{}
  \fancyhead[L,R]{\headerfmtpage{\thepage}}
  \fancyhead[L]{\headerfmt{\rightmark}}
  \fancyhead[R]{\headerfmt{\leftmark}}
  \renewcommand{\headrulewidth}{0.5pt}
  \renewcommand{\footrulewidth}{0pt}
}

% une ligne et demie (note: la page de titre commence à partir de la 1.5)
\renewcommand{\baselinestretch}{1.5}

%si LaTeX ne se divise pas bien, incluez votre mot, ou peut-être empêcher le fractionnement
\hyphenation{ditmagnooitgesplitstworden dit-woord-splitst-hier}


\begin{document}

%  Titelblad

% Opmerking: gaat uit van een \baselinestretch waarde van 1.5 (die moet
% ingesteld worden voor het begin van de document environment)
\selectlanguage{french}
\begin{titlepage}

\setlength{\hoffset}{-1in}
\setlength{\voffset}{-1in}
\setlength{\topmargin}{1.5cm}
\setlength{\headheight}{0.5cm}
\setlength{\headsep}{1cm}
\setlength{\oddsidemargin}{3cm}
\setlength{\evensidemargin}{3cm}
\setlength{\footskip}{1.5cm}
\enlargethispage{1cm}
% \textwidth en \textheight hier aanpassen blijkt niet te werken

\fontsize{12pt}{14pt}
\selectfont

\begin{center}

% \includegraphics[height=2cm]{fig/logo}

\vspace{0.5cm}

CentraleSupélec\\
Projet long du cursus Supélec\\
Encadré par : J. Tomasik et A. Rimmel

\vspace{3.5cm}

\fontseries{bx}
\fontsize{17.28pt}{21pt}
\selectfont

Dessine-moi un mouton\\
Generative Adversarial Network

\fontseries{m}
\fontsize{12pt}{14pt}
\selectfont

\vspace{.6cm}

\vspace{.4cm}

\vspace{3.5cm}

François Bouvier d'Yvoire\\
Matthieu Delmas \\
Romain Poirot \\
Paul Witz

\vspace{2cm}

Etude des réseaux de neurones en perceptrons \\ avec application au concept des Generative Adversarial Network


\vspace{1cm}

Années 2017-2018

\end{center}
\end{titlepage}

% lege pagina (!!)

% titelblad (!!)

% pas de numérotation jusqu'au sommaire
\pagestyle{empty}


% %  Voorwoord (dankwoord) en toelating tot bruikleen

\newpage

\noindent \textbf{\huge Préface}

\vspace{1.5cm}

\noindent
Texte

\addvspace{4cm}

\noindent David De Reu, mei 2002\newpage

\noindent \textbf{\huge Toelating tot bruikleen}

\vspace{1.5cm}

\noindent


\addvspace{4cm}




%!TEX root = main.tex
%  Overzichtsbladzijde met samenvatting

\newpage

{
\setlength{\baselineskip}{14pt}
\setlength{\parindent}{0pt}
\setlength{\parskip}{8pt}

\begin{center}

\noindent \textbf{\huge
Dessine-moi un mouton\\[8pt]
Generative Adversarial Network
}


\end{center}

\section*{Résumé}

% TODO: samenvatting

Résumé


\section*{Mots-clefs}

% TODO: trefwoorden

Mots-clefs

}

\newpage % strikt noodzakelijk om een header op deze blz. te vermijden


\pagestyle{fancy}
\frontmatter

\tableofcontents

% opmaak voor het eigenlijke boek; onderstaande lijnen
% weglaten als de eerste regel van een nieuwe alinea moet
% inspringen in plaats van extra tussenruimte
\setlength{\parindent}{0pt}
\setlength{\parskip}{0.5\baselineskip plus 0.5ex minus 0.2ex}
\setlength{\parskip}{1ex plus 0.5ex minus 0.2ex}

% Chapitre
\mainmatter

% insérez les chapitres ici (\includes)
%!TEX root = main.tex
% \begin{figure}[htb]
% \begin{center}
% %optional pour enlever un peu d'espace blanc de votre silhouette
% %\vspace{-.3cm}
%  %\includegraphics[keepaspectratio,width=0.5\textwidth]{fig/nomde}
% % analoog
% %\vspace{-0.6cm}
%  % \caption{ici un logo}
%  % \label{fig:ruglogo}
% %analoog
% %\vspace{-.6cm}
% \end{center}
% \end{figure}

% Utilisation du template
% \cite(reference à citer)
% \ref(figure à référencer)

\chapter{Introduction aux réseaux de neurones et premières applications}
Le première partie de ce projet a pour but de comprendre le fonctionnement des réseaux de neurones et leurs applications à la classification. Une structure informatique en python pour les utiliser sera également mise en place.

\section{Outils utilisés pour le projet}

Ce projet possède une dimension de conception logicielle. Il s'agit de programmer des réseaux de neurones efficaces et performants adaptés aux problèmes que nous souhaitons résoudre sans utiliser de bibliothèques existantes.\\
Étant donné l'évolution prévue de notre code (du perceptron simple pour les problèmes de XOR ou du MNIST, puis la mise en place du GAN, et enfin toute sortes d'améliorations utiles), nous devons être particulièrement vigilants sur la souplesse de notre code. La programmation en équipe, sur une longue durée et avec de telles contraintes, nécessite la mise en place d'outils et certains choix techniques.

\section{Réseaux de neurones et fonctionnement}


Les réseaux de neurones font partie des piliers de l'intelligence artificielle. Leur fonctionnement est basé sur une interprétation sommaire du cerveau humain. Des neurones seuls reçoivent des signaux, les traitent et renvoient un signal de sortie. Les neurones sont alors agrégés en un réseau avec des entrées et des sorties globales. On modélise la plasticité du cerveaux par des paramètres variables qui changent au cours de l'apprentissage. Celui-ci se fait en comparant les sorties de notre réseau aux résultats attendus.

Les réseaux ainsi obtenus et entrainés sont donc des fonctions. L'apprentissage permet aux réseaux d'approcher les fonctions que nous souhaitons. L'objectif est d'approcher des fonctions qui ne sont pas facilement réalisables avec les moyens usuels. Par exemple, il est très difficile de définir la fonction indicatrice des chiffres manuscrits alors qu'un réseau neuronal est capable de le faire.

\subsection{Le neurone} % (fold)
\label{sub:le_neurone}
L’unité de base du réseau est le neurone, on peut l’imaginer comme une fonction mathématique $F$. Il possède $n$ entrées (ou plutôt un vecteur $X$ de dimension $n$), chacune affectée d’un poids $w_i$ (on a donc un vecteur de poids $w$, et une fonction mathématique de $\R$ dans $\R$ non linéaire dite fonction d'activation $\sigma$. Le rôle du neurone sera de renvoyer le résultat de la fonction d'activation appliquée à la somme des entrées pondérées par leurs poids respectifs : on a $F(x) = \sigma(w^T x)$. On peut ajouter un biais $b$ comme paramètre de notre neurone, ce qui permet de donner un aspect affine au calcul : $F(x) = \sigma(w^T x + b)$

\begin{figure}[h]
  \centerline{\includegraphics[width=0.6\linewidth]{fig/schemaneurone.jpg}}
  \caption{Schéma d'un neurone}
  \label{fig:neurone}
\end{figure}

Un exemple simple du réseau de neurones est la séparation d’un plan en deux.\\
Imaginons un neurone à deux entrées $e_1$ et $e_2$, chacune attribuée d’un poids $w_1$ et $w_2$. On affecte au neurone un biais $b$ et une fonction d’activation seuil (Heavyside par exemple).\\
Notre neurone renverra : 1 si $(e_1w_1+e_2w_2) - b >0$ et 0 sinon.

Si les $e_1$ et $e_2$ représentent les abscisses et ordonnées d’un point du plan, on reconnaît dans l’argument de la fonction d’activation l’équation d’une droite affine. Notre neurone pourra donc distinguer les points du plan selon le côté de la droite où ils se trouvent.

On peut déjà voir qu’une modification des poids entraînera une délimitation différente du plan. On peut donc imaginer faire « apprendre » au réseau une délimitation choisi en modifiant ses poids. Nous reviendrons sur ce concept par la suite.\\
Cependant, les applications d’un neurone seul sont vite limitées. C’est pourquoi on va s’intéresser à en connecter plusieurs entre eux.

\subsection{Réseau de neurones et perceptron} % (fold)
\label{sub:reseau_de_neurones}
On a déjà vu que le neurone se prêtait bien à une séparation binaire des données. On va voir que l’organisation de neurones en réseau permet de meilleures classifications.\\
L’organisation du réseau se fera au moyen de couches de neurones. Une couche est un ensemble de neurones possédant les mêmes entrées. Cette couche aura alors plus d'une sortie (une par neurone dans la couche), ce qui permet de généraliser aux sorties vectorielles le concept du neurone seul. En ayant ainsi une sortie vectorielle, on peut utiliser les sorties d'une couche en tant qu'entrée d'une autre couche, ce qui complexifie encore les fonctions possiblement décrites par les neurones.
On peut imaginer des réseaux plus complexes où la sortie n'est pas réutilisée dans la couche suivante mais plusieurs couches plus loin ou dans des couches antérieures. On peut en réalité construire le graphe de neurones que l'on veut, mais, dans la pratique, seules certaines structures sont utilisés. 

\begin{figure}[h]
  \centerline{\includegraphics[width=0.5\linewidth]{fig/schemacouche.jpg}}
  \caption{Exemple d'une couche de neurones}
  \label{fig:couche}
\end{figure}

Le perceptron est un modèle de réseau de neurones auquel on va s’intéresser particulièrement. Il s'agit d'un réseau linéaire où chaque couche est entièrement connectée à la suivante, c'est-à-dire que chaque neurone d'une couche prend en entrée toutes les sorties de la couche précédente. On ne trouve aucune boucle dans le graphe d'un perceptron, c'est donc une propagation vers l'avant.
L’utilité d’avoir plusieurs couches se comprend facilement. Si on reprend notre exemple du problème de classification des points, on peut imaginer, par exemple, quatre neurones qui enverront leur sortie sur un neurone à quatre entrées. Chacun des neurones réalisera la séparation du plan en deux selon le principe déjà évoqué précédemment. Le neurone de la couche de sortie pourra réaliser facilement le rôle d’un ET logique. On vient de sélectionner un carré dans le plan. En étendant le raisonnement, on voit qu’un réseau à deux couches permet de sélectionner n’importe quelle zone convexe de l’espace des entrées (ici du plan). De même, un réseau à trois couches pourra sélectionner n’importe quelle zone concave de l’espace des entrées.

\begin{figure}[h]
  \centerline{\includegraphics[width=0.5\linewidth]{fig/espace1neurone.jpg}}
  \caption{Frontière de décision pour un perceptron simple à 1 neurone et 2 entrées}
  \label{fig:espace1neurone}
\end{figure}

On appelle la dernière couche, celle qui donne le résultat du réseau de neurones, la couche de sortie; et la première couche où l'on donne les entrées est appelée couche d'entrée. Les autres couches sont appelées des couches cachées. Cette appellation vient du fait qu’a priori, nous n’avons aucun moyen de voir ou de corriger les comportements des neurones cachés. En effet, avec la seule donnée de la sortie, l’influence des poids des couches cachées sur celle-ci n'est pas évidente. C'est ce que nous allons étudier dans la partie suivante. 
% subsection reseau_de_neurones (end)




% subsection le_neurone (end)

\subsection{Apprentissage par rétro-propagation}

Il existe plusieurs types d'apprentissages du réseau. Les deux grandes catégories sont l'apprentissage supervisé et l'apprentissage non supervisé. 
Un apprentissage supervisé nécessite une base d'apprentissage à enseigner au réseau. Elle est composée d'associations entre entrées et sorties voulues. Le réseau déduira de cette base les autres cas qu'on ne lui aura pas appris. Une apprentissage non supervisé ne nécessite pas une telle base d'apprentissage.

Dans le cadre du perceptron, nous utilisons un apprentissage supervisé, la rétro-propagation des erreurs. Il s'agit en fait d'effectuer une descente de gradient. Nous avons notre réseau qui représente $F$ une fonction, cette fonction dépend de nombreux paramètres qui sont les poids $W$ de tous les neurones de toutes les couches (ainsi que les biais). Lorsque l'on veut faire tendre $F$ vers une fonction $G$, et que l'on peut avoir une distance entre ces deux fonctions, on obtient alors un problème de minimisation, avec pour paramètres d'optimisation les poids $W$. L'algorithme de descente de gradient consiste à calculer pour chaque paramètre le gradient par rapport à la sortie (c'est une descente de Newton quand on est à une dimension) afin de déplacer ce paramètre dans la bonne direction pour minimiser la sortie.

Comme dit précédemment, on n'a pas facilement accès aux couches cachées, et il parait très coûteux d'opter pour des calculs de dérivées discrètes pour obtenir les gradients voulus pour tous les paramètres.
La rétro-propagation consiste à calculer l’influence de chaque paramètre sur la sortie et à les mettre à jour en fonction de cette influence. On a vu que les couches de neurones avaient leurs propres sorties, on peut donc calculer l'influence des poids d'une couche sur sa sortie propre, puis s'en servir pour calculer l'influence sur les entrées, qui sont la sortie d'une couche précédente, on procède comme suit : 

Les paramètres que l’on fait évoluer sont les poids et les biais.
La formule de mise à jour est la suivante :
\[
W(t+1) = W(t) + \eta \frac{\partial E}{\partial W} 
\]
avec $\eta$ le pas de convergence, $\frac{\partial E}{\partial W} $ la matrice de terme général $\frac{\partial E}{\partial W_{i,j}} $.\\
Pour pouvoir mettre à jour les poids, il faut donc calculer les $\frac{\partial E}{\partial W_{i,j}} $.\\
A la couche $k$ l'influence des poids est donnée par : 
\[
	\frac{\partial E^p}{\partial W _k} = \frac{\partial F}{\partial W}(W_k, X_{k-1})\frac{\partial E^p}{\partial X_k}
\]

Avec $\frac{\partial F}{\partial W}(W_k, X_{k-1})$ la matrice jacobienne de F par rapport à la variable $W_k$.

Pour pouvoir calculer l'influence des poids de toutes les couches, il faut donc calculer $\frac{\partial E^p}{\partial W_k}$

On peut calculer par récurrence cette valeur pour toutes les couches.
\[
	\frac{\partial E^p}{\partial X _{k-1}} = \frac{\partial F}{\partial X}(W_k, X_{k-1})\frac{\partial E^p}{\partial X_k}
\]

Avec $\frac{\partial F}{\partial X }(W_k, X_{k-1})$ la matrice jacobienne de $F$ par rapport à la variable $X_k$. De plus, dans un perceptron, on peut noter la sortie de la couche $k$ : 
\[
Y_k = W_k X_k \]
\[
X_k = F(Y_k)
\]

On obtient donc ces 3 équations : 
\begin{align*}
\frac{\partial E^p}{\partial y_k^i} &= f'(x_k^i)\frac{\partial E^p}{\partial x_k^i} \\
\frac{\partial E^p}{\partial w_k^{i,j}}&= x^j_{k-1} \frac{\partial E^p}{\partial y_k^i}\\
\frac{\partial E^p}{\partial x_{k-1}^m} &= \sum_i(w_k^{im}\frac{\partial E^p}{\partial X_k})
\end{align*}

En forme matricielle, ces équations donnent :
\begin{align*}
\frac{\partial E^p}{\partial Y_k} &= Diag(f'(x_k^i))\frac{\partial E^p}{\partial x_k^i} \\
\frac{\partial E^p}{\partial W_k}&= X^T_{k-1} \frac{\partial E^p}{\partial Y_k}\\
\frac{\partial E^p}{\partial X_{k-1}} &= W_k^T\frac{\partial E^p}{\partial X_k}
\end{align*}

On obtient donc une formule de récurrence que l'on doit propager de la dernière couche à la première couche (d'où le terme rétro-propagation). On sait donc maintenant mathématiquement faire l'apprentissage d'un perceptron : il nous faut des données connues, une fonction d'erreur (appelée Loss Function dans la littérature), et appliquer cet algorithme.


\section{Conception logicielle des réseaux de neurones}
\paragraph*{} % (fold)
L'un des objectifs du projet est la conception d'une librairie permettant l'implémentation de réseaux de neurones. Notre démarche est la suivante, nous cherchons à mettre en place la structure la plus simple possible mais également la plus souple possible. Ainsi nous ne cherchons pas l'exhaustivité de notre librairie, mais nous pouvons facilement la compléter dès lors que nous avons besoin de fonctionnalité supplémentaire.

\paragraph*{} % (fold)
Notre code est structuré autour de 3 types de classes, les classes permettant la création et le fonctionnement d'un ou plusieurs réseaux de neurones (ce sont les classes qui font l'intelligence du programme, noté $brain$), les classes apportant des outils de compréhension et de travail sur les réseaux (affichage de résultats, chargement et sauvegarde de paramètres, etc) et les classes permettant de lancer une expérience (classes $main$, instanciant les objets et les expériences).

Le projet est découpé en 2 répertoire Github, le premier correspondant au code le plus simple, fonctionnel sur le problème du XOR, le second correspondant au développement suivant. Ces derniers développement correspondent à la généralisation à tout types de problèmes d'apprentissage de perception simple, puis à la mise en place du GAN et toutes les évolutions que nous avons mis en place.

Vous pouvez retrouver les codes sur https://github.com/Supelec-GAN/Salamandre-XOR.git et sur https://github.com/Supelec-GAN/Salamandre-Code.git .

\subsection{Structure du code} % (fold)
\label{sub:structure_du_code}

Afin de pouvoir implémenter facilement les calculs matriciels obtenus plus tôt, nous définissons notre plus bas niveau d'intelligence par une classe NeuronLayer représentant une couche de neurone. \\

Une couche de neurones est définie par sa matrice de poids $weights$, son vecteur de biais $biais$ ainsi que sa fonction d'activation $activation_function$, nécessairement commune à tous les neurones dans cette structure.

\paragraph{Présentation des méthodes} % (fold)
 \label{par:presentatation_des_methodes}
 
 \begin{itemize}
 	\item $compute$ : Propagation d'une entrée au sein de cette couche
 	\item 
 \end{itemize}
 % paragraph presentatation_des_methodes (end) 

% subsection structure_du_code (end)



\section{Application au problème du XOR}

\paragraph*{}
Lorsque l'on souhaite travailler sur des algorithmes d'apprentissage par ordinateur, il est recommandé de les essayer sur des problèmes connus afin d'en vérifier les performances. \\
Le problème du XOR est l'un des plus classiques car il apporte de nombreuses difficultés.

L'objectif du XOR est de séparer le plan complexe en quatre cadrants, $(x >0, y > 0)$, $ (x>0, y<0)$, $ (x<0, y>0)$ et $ (x>0, y<0) $. Pour l'expérimentation, on restreint le plan à $[-1;1]^2$. Les sorties attendues par le réseau de neurones sont alors 1 pour les points tel que $x*y > 0 $ et -1 pour les points tels que $x*y<0$. \\
Le premier intérêt de ce problème est qu'il est non linéaire. Cela se traduit par le fait qu'une droite séparant le plan en 2 ne répond pas du tout au problème.

C'est en se basant sur la résolution du XOR que nous avons construit notre structure de réseau et vérifié la cohérence de notre code. La littérature propose comme réseau le plus simple pour ce problème une couche cachée de 2 neurones, avec 2 entrées ($x$ et $y$) et 1 sortie dans $[-1, 1]$. Nous avons étudié également quelques autres formes de réseaux pour comparer les résultats.

\paragraph{Notion de résultats} % (fold)
\label{par:notion_de_resultats}
La notion de résultats nécessite d'être correctement définie afin de pouvoir être interprétée correctement, en particulier pour la comparaison à d'autres résultats obtenus par nous-même ou par d'autres personnes. \\
La structure de perceptron sous cette forme classe les objets que l'on donne en entrée. Généralement, le résultat est défini par rapport à un pourcentage de succès dans cette classification. Pour l'obtenir, on commence par définir une erreur relative, c'est-à-dire une distance entre la sortie cible et la sortie obtenue. Un seuil est alors appliqué afin de définir une sortie booléenne de la classification de l'entrée.\\
Dans le cas du XOR, on met en place un seuil de 0.5, c'est à dire que, si l'erreur est inférieure à 50\%, le réseau a raison. Cela peut s’interpréter comme suit : le réseau donne un résultat qui indique sa confiance dans la sortie. 1 ou 0 si il est certain que la sortie doit être 1 ou 0, 0.5 si il ne peut départager l'un ou l'autre, le seuil consiste à dire que sa réponse est celle en qui il a le plus confiance.\
On cherche également à évaluer la vitesse d'apprentissage. Ainsi, on calcule le pourcentage de succès du réseau à intervalles réguliers au cours de l'apprentissage. Les réseaux étant soumis à une forte composante aléatoire (l'ordre d'apprentissage, ainsi que l'initialisation des poids), on effectue des apprentissages dans les mêmes conditions plusieurs fois afin d'obtenir des courbes moyennes, et des intervalles de confiances justifiant nos résultats.
% paragraph notion_de_résultats (end)

\paragraph{Réseau en $2\rightarrow2\rightarrow1$} % (fold)

Les résultats obtenus au début sur ce réseau extrêmement simple semblaient tout à fait aléatoires et nous ont permis de détecter des erreurs de traduction des équations de rétro-propagation en code Python. Nous avons finalement pu obtenir des résultats satisfaisants, comme le montre la figure \ref{fig:2_2_1}. 

\begin{figure}[h!]
  \includegraphics[width=\linewidth]{fig/xor221_eta009.png}
  \caption{Application d'un réseau 2-2-1 au plan pour xor}
  \label{fig:2_2_1}
\end{figure}


% paragraph réseau_en_2_2_1 (end)

\paragraph{Importance du pas d'apprentissage $\eta$}

L'un des paramètres de la descente de gradient est le pas d'apprentissage $\eta$. Il intervient dans le formule de mise à jour des poids :
\[
W(t+1) = W(t) + \eta \frac{\partial E}{\partial W} 
\]
Plus il est petit, plus l'algorithme avancera lentement. Cependant, s'il est trop grand, l'algorithme ne se rapprochera pas du minimum mais divergera. Il faut donc le choisir suffisamment petit pour que cela converge mais suffisamment grand pour que ce ne soit pas trop lent. La figure \ref{fig:2_2_2_1} montre en 10000 apprentissages le résultat d'un réseau entraîné sur XOR selon le pas d'apprentissage. On remarque notamment que, lors qu'il est trop élevé, le réseau n'arrive pas à découper l'espace.

\begin{figure}[h!]
  \centering
  \begin{subfigure}[b]{.3\linewidth}
    \includegraphics[width=\linewidth]{fig/xor2221_eta01.png}
    \caption{eta =0.1}
  \end{subfigure}
  \quad
  \begin{subfigure}[b]{.3\linewidth}
    \includegraphics[width=\linewidth]{fig/xor2221_eta05.png}
    \caption{eta =0.5}
  \end{subfigure}
  \quad
  \begin{subfigure}[b]{.3\linewidth}
    \includegraphics[width=\linewidth]{fig/xor2221_eta09.png}
    \caption{eta =0.9}
  \end{subfigure}
  \caption{Xor en 2-2-2-1 avec différents pas d'apprentissages au bout de 10000 apprentissages}
  \label{fig:2_2_2_1}
\end{figure}

\paragraph{Structure du réseau}
	La structure du réseau choisie est également importante. Comme expliqué plus haut, si le réseau est 2-2-1, il ne peut pas délimiter l'espace comme on le souhaite. Cela signifie que, pour la fonction XOR avec laquelle on travaille (à savoir qu'on représente VRAI par 1 et FAUX par -1), il ne peut pas découper le plan selon les axes des abscisses et ordonnées. En revanche, il va essayer de le délimiter de façon à accorder aux points (1,1) et (-1, -1) la valeur VRAIE et (1,-1) et (-1,1) la valeur FAUX. Pour cela, il trace une bande comme sur la figure \ref{fig:struct221}. En effet, il y a priorité sur ces valeurs puisque la base d'apprentissage de notre réseau est constituée de couples de -1 et de 1.
	
\begin{figure}[h!]
  \centering
  \begin{subfigure}[b]{.3\linewidth}
    \includegraphics[width=\linewidth]{fig/xor221_eta009.png}
    \caption{Structure 2-2-1}
    \label{fig:struct221}
  \end{subfigure}
  \quad
  \begin{subfigure}[b]{.3\linewidth}
    \includegraphics[width=\linewidth]{fig/xor241_eta016.png}
    \caption{Structure 2-4-1}
    \label{fig:struct241}
  \end{subfigure}
  \quad
  \begin{subfigure}[b]{.3\linewidth}
    \includegraphics[width=\linewidth]{fig/xor2221_eta05.png}
    \caption{Structure 2-2-2-1}
    \label{fig:struct2221}
  \end{subfigure}
  \caption{Xor avec différentes structures}
  \label{structxor}
\end{figure}

Cependant, lorsqu'on lui accorde plus de neurones, 4 neurones sur la première couche comme sur la figure \ref{fig:struct241}, ou 3 couches par exemple comme sur la figure \ref{fig:struct2221}, le réseau découpe l'espace comme souhaité. En revanche, le XOR avec la structure 2-2-2-1 est assez instable : pour les mêmes paramètres, on peut obtenir un réseau qui converge et un autre qui se plante complètement comme le montre la figure \ref{fig:2_2_2_1_instable}
	
\begin{figure}[h!]
  \centering
  \begin{subfigure}[b]{.4\linewidth}
    \includegraphics[width=\linewidth]{fig/xor2221_eta05.png}
    \caption{XOR qui converge}
    \label{fig:struct221}
  \end{subfigure}
  \quad
  \begin{subfigure}[b]{.4\linewidth}
    \includegraphics[width=\linewidth]{fig/xor2221_eta05v2.png}
    \caption{XOR planté}
  \end{subfigure}
  \caption{XOR en 2-2-2-1 avec eta =0.5}
  \label{fig:2_2_2_1_instable}
\end{figure}

\paragraph{Conclusion sur le XOR} % (fold)
\label{par:conclusion_sur_le_xor}
Pas d'apprentissage très petit par rapport à la littérature\\
Influence des fonctions d'activation
% paragraph conclusion_sur_le_xor (end)

\section{Application à la base de données MNIST}

\subsection*{Description du problème} % (fold)

Pour le problème du MNIST qui consiste à apprendre à reconnaitre des chiffres manuscrits, les données étaient les suivantes : 
\begin{itemize}
	\item 60000 images pour l’apprentissage, avec leurs étiquettes
	\item 10000 images de test
\end{itemize}

Toutes les images ont une dimension de 28*28 pixels en noir et blanc. Ces sets d’images sont récupérables sur le site http://yann.lecun.com/exdb/mnist/ sous le format IDX. L’extraction de ce format vers une liste python est faite grâce au module python-mnist.

\subsection*{Paramètres généraux utilisés :} % (fold)
Les fonctions d’activation utilisées pour toutes les expériences ici sont des sigmoïdes de paramètre $\mu$ : 
$\sigma_\mu(x) = \frac{1}{1+e^{-\mu x}}$\\
On pourra étudier l’influence de $\mu$ sur la vitesse de convergence.
Puisque les fonctions d’activation utilisées sont des sigmoïdes dont la sortie est dans [0, 1], les valeurs d’entrées situées entre 0 et 255 sont normalisées entre 0 et 1.\\
L’erreur utilisée sur la couche de sortie est l’erreur quadratique.\\
Les poids sont initialisés avec une répartition gaussienne centrée réduite. Les biais sont initialisés à 0.\\
De bons résultats ont été obtenus avec le réseau suivant, conformément à la littérature :
\begin{itemize}
	\item Eta 0.2
	\item Sigmoïde 0.1
	\item Réseau à une couche cachée de 300 neurones et 10 neurones de sortie (784-300-10)
	\item Apprentissage stochastique
\end{itemize}
Avec ce type de réseau, on obtient rapidement des taux de succès proche de 95\% après 10 passes de l’ensemble du set d’apprentissages, et un écart-type final de 0.001223411. Nous allons maintenant voir l’influence des différents paramètres.

\begin{figure}[h!]
  \includegraphics[width=\linewidth]{fig/MNIST_result_1.png}
  \caption{Courbe de réussite du réseau sur MNIST}
  \label{fig:mnist_result_1}
\end{figure}

\subsubsection*{Variation d’êta :}


Sur le réseau 300-10 précédent, une augmentation du êta de 0.2 à 10 ne semble qu’améliorer la vitesse de convergence comme le montre la figure \ref{fig:mnist_influence_eta}. 
L’écart-type n’augmente pas et on obtient une meilleure précision à la fin.

\begin{figure}[h!]
  \includegraphics[width=\linewidth]{fig/MNIST_influence_eta.png}
  \caption{Courbe de réussite du réseau sur MNIST avec différents $\eta$}
  \label{fig:mnist_influence_eta}
\end{figure}

\subsubsection*{Choix du paramètre de la sigmoïde :}
Ici, l’influence du choix de la sigmoïde est observée. Les tests ont été effectués avec $\mu = 0.1$ et $\mu = 0.5$ comme paramètre de sigmoïdes.
[[insert courbe (cf 17/01/18)]]
On remarque qu’en tout point, choisir 0.1 en paramètre à la place de 0.5 est mieux : vitesse de convergence, précision finale, écart-type. Ce résultat empire avec un êta plus élevé, au point de ne plus réussir à apprendre. On restera donc sur une sigmoïde de paramètre 0.1 pour la suite des expériences.

\begin{figure}[h!]
  \centering
  \begin{subfigure}[b]{.5\linewidth}
    \includegraphics[width=\linewidth]{fig/MNIST_inflsigm_eta02.png}
    \caption{$\eta=0.2$}
  \end{subfigure}
  \quad
  \begin{subfigure}[b]{.5\linewidth}
    \includegraphics[width=\linewidth]{fig/MNIST_inflsigm_eta07.png}
    \caption{$\eta=0.7$}
  \end{subfigure}
  \begin{subfigure}[b]{.5\linewidth}
    \includegraphics[width=\linewidth]{fig/MNIST_inflsigm_eta1.png}
    \caption{$\eta=1$}
  \end{subfigure}
  \caption{Comparaison de deux sigmoïdes pour plusieurs $\eta$}
  \label{fig:MNIST_inflsigm}
\end{figure}

\subsubsection*{Différents réseaux :} 
Intuitivement, un réseau avec plus de couches cachées devrait obtenir une meilleure précision, mais devrait avoir un temps d’apprentissage plus long. Ces résultats se confirment avec des expériences sur le réseau à une couche cachée (300 neurones) précédent, et un réseau sans couche cachée. Ce dernier converge très vite aussi bien vis-à-vis du nombre d’apprentissages nécessaire que du temps de calcul. Cependant, il est difficile de dépasser les 90\% de succès. Alors que sur le réseau avec couche cachée, on arrive à obtenir moins de 5\% d’erreur. En revanche, les temps de calculs sont plus élevés. Le réseau avec deux couches cachées, 1000 puis 300, a aussi été testé. Les résultats ici sont satisfaisants, cependant l’amélioration des résultats n’est pas très importante, alors que les temps de calculs augmentent fortement.
[[insert courbe (cf 17/01/18)]]
Ce problème est bien connu dans l'étude des réseaux de neurones, et a justifié l'apparition de l'apprentissage profond, que nous évoquerons plus tard.
Temps de calcul approximatifs pour une passe du set d’apprentissage, et un test tous les 1000 apprentissages :
\begin{itemize}
	\item 5-6 minutes pour le 784-1000-300-10
	\item 4 minutes pour le 784-300-10 
\end{itemize}
Ces temps peuvent être améliorés avec l’introduction du batch learning qui permet de calculer les résultats des tests plus rapidement.
%!TEX root = main.tex
% \begin{figure}[htb]
% \begin{center}
% %optional pour enlever un peu d'espace blanc de votre silhouette
% %\vspace{-.3cm}
%  %\includegraphics[keepaspectratio,width=0.5\textwidth]{fig/nomde}
% % analoog
% %\vspace{-0.6cm}
%  % \caption{ici un logo}
%  % \label{fig:ruglogo}
% %analoog
% %\vspace{-.6cm}
% \end{center}
% \end{figure}

% Utilisation du template
% \cite(reference à citer)
% \ref(figure à référencer)

%\bibliography{reference}

\chapter{Generative Adversarial Networks}

\paragraph{}
La première partie de cette étude nous a permis de maîtriser l'utilisation de réseaux en perceptron et de structurer une architecture logicielle efficace et souple pour l'étude des GAN. \\
Après avoir obtenu des résultats satisfaisants dans la classification de motif sur la base MNIST, nous étudions la génération de données à l'aide de réseaux de neurones en nous basant sur le concept de GAN, introduit par I. Goodfellow en 2016 \cite{nips-2014}.\\
Notre objectif dans cette partie est d’appréhender le concept de GAN et de l'appliquer sur notre programme afin d'étudier les différents paramètres. 
\section{Principe}
\paragraph{}
Le GAN s'inscrit dans les problèmes de générations de données par ordinateurs. Ses modèles cherchent à produire de données nouvelles respectant un certain nombre de contraintes. Les applications possible sont très nombreuses, tant au niveau scientifiques qu'industrielles, avec par exemple la modélisation de nouvelles protéines, le dessin de circuit intégrés, etc. 
Le but de nos GAN sera de générer des images que l’on ne pourra distinguer de « vraies » images, prises avec un appareil photo.\\

Le principe général est le suivant : \\ un GAN est constitué de deux réseaux de neurones, le Générateur (G) et le Discriminateur (D). Le Générateur a pour but de créer les images et le Discriminateur de déterminer si les images qu’on lui donne sont de « vraies » images ou ont été créées par le Générateur. Ces deux réseaux sont mis en compétition : le Générateur a pour but de tromper le Discriminateur tandis que le Discriminateur doit détecter les « fausses » images.\\
L’apprentissage du Discriminateur se fait à la fois sur des images générées par le Générateur et de « vraies » images, issues d’une banque d’images afin de continuellement améliorer sa capacité de discernement.\\
L’apprentissage du Générateur dépend de la réponse du  Discriminateur : lorsqu’il génère une image, on la donne au Discriminateur pour voir si le Générateur a réussi à le tromper. Le discriminateur sert donc de fonction d'erreur au Générateur.

\paragraph{}
De façon plus formelle, on travaille avec 3 distributions : $p_x$, la distribution idéale des vraies images, $p_{data}$, la distribution de l'échantillon des vraies images et $p_{model}$ la distribution réalisée par les images issues du Générateur. Le but de l’apprentissage est de rapprocher $p_{model}$ de $p_x$. Comme $p_x$ nous est inconnu, on va plutôt s'approcher de $p_{data}$.\\
On dispose de deux fonctions de coûts $J_D(\theta_G, \theta_D)$ et $J_G(\theta_G, \theta_D)$, représentant respectivement les fonctions de coûts du Discriminateur et du Générateur. On note $\theta_G$ et $\theta_G$ les paramètres des réseaux. Les fonctions de coûts dépendent bien des paramètres des deux réseaux car le Discriminateur apprend à discerner les vrais images des fausses, dépendantes du générateur, et le générateur apprend via le résultat du Discriminateur.\\ C'est un problème d'optimisation simultanée. 
Il peut également être décrit comme un problème de jeux à informations complètes. G à accès aux données de D, mais ne peut influer que sur $\theta_G$ et D à accès aux données de G, mais ne peut influer que sur $\theta_G$. Cette vision permet de déduire un algorithme ou chaque joueur va faire un mouvement de manière optimal, afin de tendre vers un équilibre de Nash.

\section{Apprentissage}

\paragraph{}
L’apprentissage consiste à appliquer cette méthode de jeu à l'apprentissage des réseaux de neurones.
Nous fournissons au Discriminateur des images $x$ (de la BDD MNIST par exemple) et lui demandons de nous renvoyer un réel entre 0 et 1, qui représente son degré de confiance sur le fait que l’image fournie ai été tirée d’une banque de donnée authentique ou du Générateur. Les réponses attendus sont respectivement ($D(x) = 1$) et ($D(x) =0$) ce qui nous permet de calculer des erreurs pour la descente de gradient du Discriminateur.\\ 
Le Générateur, quant à lui, génère une image à partir d’un vecteur de bruit $z$. Cette image est ensuite jugée par le Discriminateur : $D(G(z)) = 1$ si le Générateur a dupé le Discriminateur et 0 sinon. L'objectif du Générateur est d'être le plus proche possible de la première situation, l'erreur pour la descente du gradient du Générateur en est déduite.\\
L'apprentissage complet se fait en alternant les 2 phases successivement, chaque réseau jouant tour à tour. On parle de réseau concurrent car le Discriminateur cherche à obtenir $D(G(z)) = 0 $ pour tout $z$ et le Générateur $D(G(z)) = 1$.

 \section{Paramètres des GANS}

 Voici une description des paramètres principaux sur lequel on peut jouer pour l'implémentation d'un GAN. Ils sont nombreux car la description précédente est en réalité peu restrictive.

 \begin{itemize}
 	\item Fonctions de coûts
 	\item Ratios d'apprentissage
 	\item Paramètres classiques des réseaux de neurones (Structures des réseaux, pas d'apprentissage, etc.)
 \end{itemize}

\section{Structure et utilisation du code} % (fold)
{description des modifications importantes pour le GAN (or optimisation du code source)}

\section{Premier résultats pour des GANs simples}
{présentation des paramètres principaux (bruit seulement à l'entrée, etc)}
{description des résultats avec des GANs simple, sans optimisations, et évaluation des différences de paramètres}

\section{Mode Collapse et Bruit en entrée}

{Le mode collapse aura été présenté à la section d'avant, tentative d'explication de pourquoi on en sort avec le bruit}

\section{Résultat sans collapse}
{même chose que la section premiers résultats, mais avec le bruit en entrée}

% Fin du chapitre, les autres améliorations seront dans d'autres chapitres.
%!TEX root = main.tex
% \begin{figure}[htb]
% \begin{center}
% %optional pour enlever un peu d'espace blanc de votre silhouette
% %\vspace{-.3cm}
%  %\includegraphics[keepaspectratio,width=0.5\textwidth]{fig/nomde}
% % analoog
% %\vspace{-0.6cm}
%  % \caption{ici un logo}
%  % \label{fig:ruglogo}
% %analoog
% %\vspace{-.6cm}
% \end{center}
% \end{figure}

% Utilisation du template
% \cite(reference à citer)
% \ref(figure à référencer)

\chapter{Améliorations classiques des Réseaux de neurones appliqués à GAN}


\paragraph{}
Les réseaux de neurones que nous utilisons pour le GAN sont de simples perceptrons. De nombreuses méthodes pour améliorer les résultats et/ou la convergence ont été proposés pour ces types de réseaux. Nous avons étudié en particulier les algorithmes de descente de gradient avec pas adaptatif et les réseaux de neurones à convolution. 

\section{Algorithmes de descente de gradient à pas adaptatif}
\paragraph{}
En utilisant la descente de gradient classique, nous avons constaté que, dans le générateur, presque seule la couche de sortie travaillait. Le GAN ne générait alors pas des images d'assez bonne qualité ni assez diverses. Nous nous sommes donc intéressés à d'autres algorithmes de descente, dans l'espoir qu'ils soient plus efficaces et atteignent plus "en profondeur" les réseaux. 
\paragraph{}
Les algorithmes de descente auxquels nous nous sommes intéressés sont notamment des algorithmes à pas adaptatif. En effet, dans ces algorithmes, le pas change au fur et à mesure de l'apprentissage. Il peut être grand au début, pour aller dans la bonne direction, et petit à la fin pour plus de précision. 

\subsection{Momentum}
Dans une descente de gradient classique, la formule de mise à jour des poids est la suivante. 
\[W_{k+1} = W_k - \eta * \frac{\partial J}{\partial W}\]
On peut également le noter :
\[\Delta W_k = -\eta*\frac{\partial J}{\partial W}\]

Une première méthode qui est compatible avec tous les algorithmes suivants est de rajouter une inertie au gradient, ou momentum. Le but est de limiter les oscillations "inutiles" qui peuvent arriver lors d'une descente de gradient. On a alors : 
\[\Delta W_k  = \mu * W_k - \eta*\frac{\partial J}{\partial W}\]

\subsection{AdaGrad}
AdaGrad (qui signifie Adaptative Gradient) est un algorithme où le pas change en fonction de l'erreur. 
On calcule la somme des carrés des gradients : 
\[g_{k+1} = g_k + (\frac{\partial J}{\partial W})^2\]

La formule de mise à jour des poids est alors : 
\[\Delta W_k = -\frac{\partial J}{\partial W}*\frac{\eta}{\sqrt{g_{k+1}}+\epsilon}\]

$\epsilon$ est une valeur arbitrairement faible pour éviter une division par zéro et pour initialiser l'algorithme. L'inconvénient majeur de AdaGrad est que la quantité $g_k$ ne peut qu'augmenter dans le temps, ce qui implique que le pas devient de plus en plus faible. Si l'apprentissage dure trop longtemps, les poids ne bougeront presque plus à cause du faible pas.

\subsection{RMSProp}
RMSProp est sensiblement identique à Adagrad mais avec une amélioration : au lieu de considérer la somme des carrés des gradients, on considère une pondération de cette somme. Cela permet de donner plus d'importance aux derniers gradients. 
On calcule donc : 
\[g_{k+1} = \gamma*g_k + (1-\gamma)*(\frac{\partial J}{\partial W})^2\]

La formule de mise à jour des poids est donc identique à celle d'Adagrad : 
\[\Delta W_k = -\frac{\partial J}{\partial W}*\frac{\eta}{\sqrt{g_{k+1}}+\epsilon}\]

Dans le calcul de $g_k$, il y a un terme quadratique. On appelle donc $g_k$ \textbf{moment d'ordre 2}. Il existe également un moment d'ordre 1 qui se calcule par :
\[g_{k+1} = \gamma*g_k + (1-\gamma)*\frac{\partial J}{\partial W}\]
Cela donne comme équation de mise à jour des poids : 
\[\Delta W_k = -\frac{\partial J}{\partial W}*\frac{\eta}{\sqrt{g_{t+1}-(m_{t+1})^2}+\epsilon}\]


\subsection{Adadelta}
Cet algorithme est similaire à RMSProp et utilise également une somme mobile pour calculer le moment d'ordre 2 du gradient, $g_k$. Cependant, au lieu d'avoir un $\eta$ fixe, on introduit $x_k$, le moment d'ordre 2 de $\Delta W_k$.
\[g_{k+1} = \gamma*g_k + (1-\gamma)*(\frac{\partial J}{\partial W})^2\]
\[x_{k+1} = \gamma*x_k + (1-\gamma)*(\Delta W_k)^2\]

On obtient donc :
\[\Delta W_k = -\frac{\partial J}{\partial W}*\frac{\sqrt{x_k + \epsilon}}{\sqrt{g_{t+1}-(m_{t+1})^2}+\epsilon}\]
\subsection{Adam}
Adam (pour Adaptive Moment Estimation) adapte le pas en fonction des moments d'ordre 1 et 2 du gradient. Notons $m_k$ le moment d'ordre 1 et $v_k$ le moment d'ordre 2.
On les calcule par :
\[m_{k+1} = \beta_1*m_k + (1-\beta_1)*g_k\]
\[v_{k+1} = \beta_2*v_k + (1-\beta_2)*g_k^2\]

Quand $m_k$ et $v_k$ sont initialisés à 0, ils sont biaisés vers 0.
Pour pallier à cela, on considère $\widehat{m_k}$ et $\widehat{v_k}$ : 
\[\widehat{m_{k}} = \frac{m_k}{1-\beta_1^k}\]
\[\widehat{v_{k}} = \frac{v_k}{1-\beta_2^k}\]

On met à jour les poids avec :
\[\Delta W_k = -\frac{\eta}{\sqrt{\widehat{v_k}}+\epsilon}*\widehat{m_k}\]

\subsection{Comparaison des algorithmes sur MNIST}

	Sur la figure \ref{fig:comp_algos}, on voit que ces algorithmes améliorent le résultat sur MNIST autant en précision qu'en vitesse de convergence par rapport à l'algorithme de descente de gradient classique. Le meilleur algorithme à utiliser est Adam avec $\eta = 0.01$, $\gamma_1=0.9$ et $\gamma_2 = 0.999$ comme paramètres. Cependant, comme l'a étudié le groupe Couleuvre, il peut être avantageux d'utiliser deux algorithmes différents sur le générateur et le discriminateur. En effet, utiliser Adam sur le générateur et RMSProp sur le discriminateur permet de "ralentir" le discriminateur qui a tendance à devenir meilleur que le générateur. 
	
\begin{figure}[ht!]
  \includegraphics[width=\linewidth]{fig/comparaisonAlgos.png}
  \caption{Comparaison des algorithmes de descentes}
  \label{fig:comp_algos}
\end{figure}

\section{Réseaux à convolution: DCGAN}

Les perceptrons ne tiennent pas compte du type de données en entrées. 
Les réseaux à convolution sont eux plus adaptés au traitement d'images et à la reconnaissance de motifs.
Nous avons donc essayer d'implémenter ce type de réseau.

\subsection{Principe}
Comme le nom l'indique, ces réseaux consistent à appliquer différentes convolutions de matrices sur les images afin d'extraire des motifs tels que des contours, des coins, des changements de couleurs, etc.
\\
Une image est dans ce type de réseau est de dimension $p \times n \times n$, avec $p$ le nombre de canaux (aussi appelé channels ou encore feature maps) de l'image.
Le filtre d'une couche est de dimension $q \times p \times m \times m$, avec $p$ le nombre de canaux en entrée et $q$ le nombre de canaux en sortie.
Un canal contient généralement un type d'information telles que les valeurs RVB de l'image, un contour, un motif, etc.
\\
Mathématiquement, on a alors l'équation de propagation suivante :
\[x^{l}_{i,j}(k) = \sum^{p-1}_{r=0}{\sum^{m-1}_{a=0}{\sum^{m-1}_{b=0}{w_{a,b}^{l,r}y^{r}_{(i+a),(j+b)}(k-1)}}},\]
où $l\in [0, q-1]$ et $(i,j)\in [0, n-m+1]^2$.
\\
On obtient alors les équations de rétropropagation suivantes :
\[\frac{\partial E}{\partial w_{a,b}^{l,r}} = \sum^{n-m}_{i=0}{\sum^{n-m}_{j=0}{\frac{\partial E}{\partial x^{l}_{i,j}(k)}y^{r}_{(i+a),(j+b)}(k-1)}}\]
\[\frac{\partial E}{\partial x_{i,j}^{l}(k)} = \frac{\partial E}{\partial y_{i,j}^{l}(k)}\sigma ' (x^{l}_{i,j}(k))\]
\[\frac{\partial E}{\partial y_{i,j}^{l}(k-1)} = \sum^{q-1}_{r=0}\sum^{m-1}_{a=0}{\sum^{m-1}_{b=0}{\frac{\partial E}{\partial x^{r}_{(i-a),(j-b)}(k)}w^{r,l}_{a,b}(k)}} \]

\subsection{Paramètres des convolutions}
Nous avons vu qu'un filtre est défini par sa surface, et son nombre de canaux en sortie, l'image en entrée imposant les canaux en entrée.
Le nombre de canaux en sortie permet de choisir le nombre de motifs que l'on veut que le réseau soit capable d'identifier.
La surface du filtre permet d'influencer sur les motifs reconnus.
On obtient alors un tenseur de poids pour la couche.
Cependant, il existe d'autres paramètres configurables pour les couches à convolutions.
Ainsi les strides et le zéro-padding sont des paramètres souvent utilisés.
\begin{itemize}
	\item Le zéro-padding consiste à ajouter des zéros autour de l'image.
	Cela peut permettre de donner un peu plus d'importance aux bords de l'image, et permet aussi d'être moins restreint sur la dimension des canaux en sortie du filtre.
	Les dimensions de l'image ne diminuent alors pas nécessairement.

	\item Les strides consiste à faire varier le pas de déplacement du filtre sur l'image, qui vaut 1 par défaut.
	Ce pas variable permet d'adapter la quantité d'information à calculer et à conserver.
\end{itemize}
Pour le choix de la fonction d'activation, la fonction ReLu, ou ses variations telles que PReLu ou LeakyReLu, est souvent conseillée. Cependant, il est nécessaire de mettre une couche de BatchNorm derrière une telle fonction, afin d'éviter que la sortie de la couche diverge trop fortement.
\\
De plus, une couche de pooling est généralement rajoutée après chaque couche de convolution.
Ce pooling permet de concentrer l'information pour mieux l'extraire ensuite.
Différents types de pooling existent tels que le max pooling ou la moyenne.

\subsection{Implémentation}
La mise en place des couches de convolutions a demandé un important travail de réorganisation du code malgré les efforts initiaux pour avoir une base souple et modulaire.
Cette nouvelle implémentation s'inspirent de frameworks tels que TensorFlow, et permet d'enchainer n'importe quel type de couche.
\\
Cependant, bien que cette implémentation tourne à la fois pour les perceptrons classiques et les couches à convolutions, nous n'avons pas réussi à obtenir de résultats convaincants sur la base MNIST ou CIFAR10 avec ces nouvelles couches.
Nous avons alors essayé de reproduire certains réseaux de la littérature, afin de comparer les résultats.
Cependant les réseaux à reproduire étaient assez gros, et ne pouvaient pas terminer en temps raisonnable sur notre implémentation des convolutions qui n'était pas extrêmement optimisée et performante.
De plus, ces réseaux utilisaient souvent des techniques d'apprentissage que nous n'avons pas, tel que le dropout.
\\
Ces différents éléments ne nous ont pas permis d'obtenir des résultats satisfaisants.
 


\include{chapitre4}
%!TEX root = main.tex
% \begin{figure}[htb]
% \begin{center}
% %optional pour enlever un peu d'espace blanc de votre silhouette
% %\vspace{-.3cm}
%  %\includegraphics[keepaspectratio,width=0.5\textwidth]{fig/nomde}
% % analoog
% %\vspace{-0.6cm}
%  % \caption{ici un logo}
%  % \label{fig:ruglogo}
% %analoog
% %\vspace{-.6cm}
% \end{center}
% \end{figure}

% Utilisation du template
% \cite(reference à citer)
% \ref(figure à référencer)

\chapter{Axes de Recherches : WGAN}

\paragraph*{} Avec les différents résultats obtenus par nos premiers GAN, nous avons pu tirer, entre autres, deux conclusions importantes. Le GAN manque cruellement de stabilité (par exemple un petit changement de paramètre l'empêche de converger correctement) et de métriques pertinentes, c'est à dire que les scores des générateurs et des discriminateurs n'ont pas d'interprétations en termes de progrès de la qualité d'image perçue.\\
Les chercheurs se sont beaucoup attardés depuis 2016 sur la première question, en comparant par exemple les différents optimiseurs possibles \cite{optimiser}, le deuxième point est lui moins souvent abordé. \\
L'article de 2017 Wasserstein GAN \cite{wgan} propose une méthode qui, en s'éloignant légèrement de la philosophie originale du papier de Goodfellow \cite{Goodfellow-et-al-2016}, tente d'apporter une réponse à ces deux questions, avec, en particulier, une métrique pertinente.

\section{Problématique de la descente de gradient simultanée}

L'article de Goodfellow semble démontrer la convergence du système GAN, cependant la mise en œuvre montre que cette convergence n'est pas aussi évidente à obtenir. En effet, il semblerait que la stratégie de descente de gradient de l'algorithme de GAN ne permette pas d'assurer cette convergence. Le blog inFERENCe \cite{conservative-field} décrit une partie du problème en se basant sur l'article The Numerics of GANs \cite{numerics-gan}.

Ces articles montrent que la descente de gradient simultanée n'est pas simplement une double descente de gradient, mais une descente de gradient dans un champ vectoriel. L'algorithme de GAN effectue l'optimisation suivante : 
\[x_{t+1} \leftarrow x_t + h v(x_t) \text{ avec } v(x) = \left(\begin{matrix}\frac{\partial}{\partial\theta}f(\theta, \phi)\\\frac{\partial}{\partial\phi}g(\theta, \phi)\end{matrix}\right).\]

$f$ et $g$ étant respectivement les fonctions de coût du Discriminateur et du Générateur. Cependant on constate 2 problèmes. D'une part, l'algorithme basé sur la théorie des jeux, qui consiste à faire "jouer" tour à tour le Discriminateur et le Générateur pour optimiser ses paramètres ne consiste qu'en une approximation de la simultanéité de la descente. D'autre part, il n'y a aucune preuve que le champ vectoriel $x$ dans lequel l'on se déplace possède des propriétés conservatives. En particulier, rien ne garantit que le rotationnel soit nul, ce qui implique que la descente de gradient ne soit pas garantie d'aller vers un minimum, même local (Figure : \ref{fig:vector_field})!

\begin{figure}[ht!]
  \centering
    \includegraphics[width=10cm]{fig/vector_field}
    \caption{champ vectoriel non conservatif : exemple de descente de gradient}
    \label{fig:vector_field}
\end{figure}

On est donc à la recherche d'une autre approche qui contournerait ce problème.



\section{L'approche Wasserstein GAN}

Le papier Wasserstein GAN \cite{arjovsky_wasserstein_2017}, propose une autre approche que celle de la théorie des jeux.

Il s'agit de calculer une divergence entre distributions afin de se servir de cette métrique pour faire l'apprentissage de l'une sur l'autre. Au premier abord, cette approche ne semble pas si éloignée de l'approche de Goodfellow, mais elle en est sensiblement différente. Dans l'approche précédente, on tente de minimiser la divergence entre deux distributions par un algorithme utilisant 2 réseaux de neurones. Cependant, nous n'avons jamais accès à cette divergence (que l'on utilise des fonctions d'erreur pour approcher la KL-divergence ou une autre), et comme nous l'avons montré précédemment, la convergence n'est pas réellement assurée. 

L'idée du papier Wasserstein GAN est de calculer explicitement une divergence entre deux ensembles. Cela n'est pas une question simple, comme nous avons pu le voir au chapitre 1. En effet, nous n'avons généralement pas accès à la distribution $p_{\text{réel}}$ mais uniquement à un échantillon tiré de cette distribution. \\ Cependant il apparait possible de calculer une divergence entre deux ensembles à l'aide des réseaux de neurones. Cela est possible en tout cas avec la divergence de Wasserstein, comme le montre ce papier, nous allons revenir sur les étapes de raisonnement.

L'objectif est de rapprocher 2 distributions en utilisant la métrique de Wasserstein à l'ordre 1. Celle-ci s'écrit :
\[
W(\mathbb{P}_r, \mathbb{P}_g)= \underset{\gamma \in (\mathbb{P}_r, \mathbb{P}_g)}{\text{inf}} \mathbb{E}_{(x,y)\sim\gamma} \left[ ||x-y||\right]
\]
On peut la nommer également distance Earth-Mover, c'est à dire distance du déplacement de terre. En effet cette métrique calcule l'effort à faire pour passer d'une distribution à l'autre (Figure : \ref{fig:E-M_distance}). Par exemple, si l'on a deux terrains contenant des tas de terre, la hauteur de terre en un point représente la densité de probabilité à cette endroit, le volume complet de terre étant le même (cela représente l'intégrale sur le terrain), alors la distance EM entre les deux terrains est l'effort minimal que l'on peut faire pour déplacer la terre de l'un des terrains pour le faire ressembler à l'autre. Il y a une infinité de façons de déplacer la terre, avec des efforts différents. La distance de Wasserstein est le coût en utilisant le plan de transport optimal. Cela se traduit également en termes de probabilités jointes, pour une approche plus mathématique. 

\begin{figure}[ht!]
  \centering
    \includegraphics[width=10cm]{fig/Earth_mover_distance}
    \caption{Illustration de la divergence Earth-Mover}
    \label{fig:E-M_distance}
\end{figure}


L'idée d'utiliser cette divergence de Wasserstein provient des propriétés mathématiques qui lui sont propres. Cette divergence est théoriquement plus pertinente pour l'apprentissage des GANs. Le détail se trouve dans ce papier, mais cela se peut se résumer ainsi : Wassertein donne plus d'informations que la KL-Divergence et ses dérivées (Jensen-Shannon, etc) en étant, entre autres, bien définie lorsque les supports de distribution sont disjoints et en étant moins souvent constante, avec donc des gradients non nuls et plus adaptés à l'apprentissage.

On ne peut toujours pas se servir de cette divergence, mais un théorème (la dualité Kantorovich-Rubinstein \cite{optimal-transport}) permet d'obtenir une nouvelle forme de cette divergence :
\[
W(\mathbb{P}_r, \mathbb{P}_g) = \underset{||f||_L<1}{\text{sup}}\mathbb{E}_{x\sim\mathbb{P}_r}\left[f(x)\right] - \mathbb{E}_{x\sim\mathbb{P}_g}\left[f(x)\right]
\]
Vous pouvez en voir une preuve simplifiée sur le blog de Vincent Hermann \cite{preuve-wgan}.
On se retrouve alors à calculer un sup sur un ensemble de fonctions (les fonctions 1-Lipschitziennes), et cela est pratique car c'est justement ce que permet de faire un réseau de neurones : Simuler des fonctions que l'on optimise par rapport à un paramètre ! On construit un réseau qui joue le rôle des fonctions f, on a donc des paramètres $W$ à optimiser pour trouver la valeur maximum. Il ne reste qu'à s'assurer que les réseaux de neurones peuvent garantir le caractère 1-Lipschitzien.

Une méthode pour s'assurer de cette propriété est de restreindre les poids dans un intervalle [-c, c] (on parle de weight-clipping). Cependant on obtient un caractère K-Lipschitzien, avec un K inconnu. (En terme de preuve, il suffit de voir que des matrices avec ses propriétés sont toutes K-Lipschitziennes avec un même K.) Cela nous assure que l'on peut calculer non pas $W(\mathbb{P}_r, \mathbb{P}_g)$ mais $K*W(\mathbb{P}_r, \mathbb{P}_g)$, mais le K étant fixe tout au long de l'apprentissage cela reste une métrique pertinente. 

Nous avons donc la possibilité avec un réseau de neurones de calculer la métrique de Wasserstein et nous allons pouvoir nous en servir pour l'apprentissage d'un générateur.

\section{Mise en œuvre}

Fort de cette métrique que l'on peut calculer, nous allons pouvoir mettre en place un algorithme d'apprentissage pour la génération d'image.

Tout d'abord, nous avons toujours besoin de deux réseaux, l'un pour la génération, dont le rôle est strictement identique aux générateurs qui ont pu être vus avant, et l'autre pour le calcul de la divergence de Wasserstein. Ce second réseau est appelé critique (plutôt que Discriminateur) par la littérature. Attention, le critique simule une fonction $f$ qui est telle que $\mathbb{E}_{x\sim\mathbb{P}_r}[f(x)] - \mathbb{E}_{x\sim\mathbb{P}_g}[f(x)] $ soit la sortie d'un batch contenant de vraies images et des images de synthèses, c'est à dire que $f(x)$ et $f(G(z))$ n'ont pas de sens contrairement aux $D(x)$ et $D(G(z))$ vus précédemment. 
Afin que le critique calcule effectivement $W(\mathbb{P}_r, \mathbb{P}_g)$, il est nécessaire de faire converger les paramètres du critique pour obtenir $f_{\text{max}}$. Une fois $W(\mathbb{P}_r, \mathbb{P}_g)$ obtenu, on peut chercher à le minimiser pour faire progresser le générateur, on effectue donc une descente de gradient à partir de ce coût. Cependant on a en fait $f_{\text{max}}(\mathbb{P}_r, \mathbb{P}_g, x)$, c'est à dire que $f_{\text{max}}$ dépend des distributions à un instant donné, par conséquent après une itération on obtient:
 \[\mathbb{E}_{x\sim\mathbb{P}_r}[f_\text{max}((\mathbb{P}_r, \mathbb{P}_g), x)] - \mathbb{E}_{x\sim\left(\mathbb{P}_g+\Delta\mathbb{P}_g\right)}[f_\text{max}((\mathbb{P}_r, \mathbb{P}_g), x)] \]

Ce qui est différent de :
 \[ \mathbb{E}_{x\sim\mathbb{P}_r}[f_\text{max}((\mathbb{P}_r, \left(\mathbb{P}_g+\Delta\mathbb{P}_g\right)), x)] - \mathbb{E}_{x\sim \mathbb{P}_g}[f_\text{max}((\mathbb{P}_r, \left(\mathbb{P}_g+\Delta\mathbb{P}_g\right)), x)] = W(\mathbb{P}_r, \mathbb{P}_g+\Delta\mathbb{P}_g)\]

 Il faut donc à chaque itération refaire converger le critique pour garantir que l'on a bien la fonction $f_{\text{max}}$ correspondant à la distribution $\mathbb{P}_g$ courante (on a en effet $\mathbb{P}_r$ fixe tout du long).

 L'algorithme se dessine alors simplement :
 \begin{itemize}
 \item Faire converger le critique en maximisant la quantité $\mathbb{E}_{x\sim\mathbb{P}_r}[f(x)] - \mathbb{E}_{x\sim\mathbb{P}_g}[f(x)] $
 \subitem - Récupérer un batch d'images réelles
 \subitem - Générer une batch d'images virtuelles
 \subitem - Calculer $\mathbb{E}_{x\sim\mathbb{P}_r}[f(x)] - \mathbb{E}_{x\sim\mathbb{P}_g}[f(x)] $
 \subitem - Rétro-propager dans le critique
 \subitem - Recommencer jusqu'à convergence
 \item Faire évoluer le générateur en minimisant la quantité $\mathbb{E}_{x\sim\mathbb{P}_r}[f(x)] - \mathbb{E}_{x\sim\mathbb{P}_g}[f(x)] $
 \subitem - Générer un batch d'images virtuelles
 \subitem - Calculer $- \mathbb{E}_{x\sim\mathbb{P}_g}[f(x)] $
 \subitem - Il n'est pas nécessaire d'effectuer le calcul pour de vraies images, car elles ne servent pas dans la rétro-propagation dans le générateur
 \subitem - Rétro-propager dans le critique sans le modifier
 \subitem - Rétro-propager dans le générateur
 \item Recommencer les 2 étapes jusqu'à convergence du générateur
 \end{itemize}

 On notera dans cet algorithme deux zones floues qui sont les ré-itérations "jusqu'à la convergence". En effet il est particulièrement difficile de s'assurer que l'on a bien convergé, ou plutôt de savoir que l'on a convergé suffisamment. Comme dans tous les algorithmes de ce genre, il y a un trade-off entre le temps de calcul et la précision.

 Pour la convergence du critique, l'article Wasserstein \cite{arjovsky_wasserstein_2017} suggère de le faire converger de façon certaine au début, i.e de s'assurer que l'on calcule bien $W ( \mathbb{P}_r, \mathbb{P}_g ) $ avec $\mathbb{P}_g$ étant la sortie du générateur avant tout apprentissage, c'est-à-dire à priori un bruit. Pour cela, on effectue longuement la première étape avant toutes choses. Puis on peut estimer qu'une itération de descente de gradient sur le générateur change très peu ce dernier, et par une hypothèse de continuité sur la distance de Wasserstein (qui semble justifiée si l'on considère cette distance comme la distance Earth-Mover) on peut estimer qu'il faudra peu d'itérations de montée de gradient pour faire de nouveau converger "suffisamment" le critique.

 Pour la convergence du générateur, on en revient aux problèmes classiques du GAN qui consiste à se demander à quel moment le générateur est suffisamment performant. Nous y reviendrons dans les parties suivantes.

 \section{Essais pratiques du WGAN}
 	Après l'étude théorique des Wassertstein GAN, nous avons mis en pratique dans notre bibliothèque python l'algorithme correspondant.

 \subsection{Modification logicielle}
 	Comme cela a été observé, si la réflexion est très différente du GAN classique, l'algorithme des WGANs reste très proche de l'algorithme original. \\
 	Nous avons dû cependant rajouter plusieurs features à notre librairie pour pouvoir en faire l'implémentation correcte.

 	\begin{itemize}
 		\item Nous avons créé une nouvelle classe \emph{WGanGame} héritant de la classe \emph{GanGame} afin de redéfinir quelques fonctions :
 			\subitem \emph{play\_and\_learn} -- Il n'y a que 2 types d'apprentissages possibles, le critique et le générateur.
 			\subitem \emph{critic\_learning} -- Remplace discriminator\_learning afin que l'entrée soit un batch d'images réelles et virtuelles, il faut également effectuer une montée de gradient.
 			\subitem \emph{generator\_learning}  -- Il est nécessaire de vérifier que le critique est en mode descente de gradient désormais.
 		\item Il a fallu mettre en place le choix de descente ou de montée de gradient au sein des réseaux.
 		En effet on ne peut pas simplement choisir un pas négatif pour le critique, car il y a une rétro-propagation décroissante dans le discriminateur lors de l'apprentissage du générateur. Cela a impliqué plusieurs gros changements dans les paramètres à passer aux fonctions de rétro-propagation.
 		\item Afin de garantir le caractère lipschitzien, il a fallu ajouter une possibilité de weight-clipping. Cela a été effectué en utilisant la méthode \emph{clamp} de numpy, et en rajoutant un paramètre au constructeur des couches FCs.
 		\item Il a fallu ajouter également 2 fonctions d'erreurs pour le critique et le générateur.

 	\end{itemize}
 \subsection{Méthodologie et résultats du WGAN}
 	Afin de vérifier si la méthode fonctionne, nous avons eu 2 approches : Reproduire les réseaux présents dans la littérature sur le WGAN et tester avec les mêmes paramètres ou appliquer le WGAN à des structures proches des meilleurs résultats que nous avons obtenus avec des GANs simples. 

 	Pour essayer de reproduire la littérature, nous nous sommes heurtés au problème suivant, elle utilise quasi-systématiquement des réseaux convolués. C'est logique car ce sont ceux qui offrent les meilleures performances lorsque les réseaux traitent des images. Cependant notre implémentation des réseaux à convolutions n'a jamais été suffisamment satisfaisante pour pouvoir s'en servir sur les GANs.

 	La deuxième méthode a eu plus de succès. Il n'est pas possible de reprendre les structures telles quelles. En effet, nous utilisions que des Sigmoids pour fonctions d'activations, restreignant les sorties. Or, pour le Wasserstein GAN, il est important de calculer librement la distance EM. C'est pourquoi le critique utilise désormais des fonctions ReLU d'activation. On a donc un critique de la forme [784-20-1], avec des poids restreints à [-0.1, 0.1]. Pour le générateur, la même structure peut être utilisée, soit un [100-300-784] pour le moment sans bruit. Il n'est pas nécessaire (ni même conseillé) de restreindre les poids du générateur.

 	Pour le reste des paramètres, nous avons suivi le papier \cite{arjovsky_wasserstein_2017}, c'est à dire un ratio de 5 pour le critique, des pas de 0.05 (au lieu de 0.0005), en utilisant RMSprop.

 	\begin{figure}[ht!]
  \centering
  	\begin{subfigure}[b]{.3\linewidth}
    \includegraphics[width=\linewidth]{fig/Wgan_result_1}
    \caption{Résultat du WGAN pour 27 000 parties, D(x) = 7.58)}
    \label{fig:Wgan_result_1_nombres}
\end{subfigure}
\quad
\begin{subfigure}[b]{.6\linewidth}
  \centering
    \includegraphics[width=10cm]{fig/Wgan_result_1_courbe}
    \caption{Distance de Wasserstein pendant l'apprentissage}
    \label{fig:Wgan_result_1_courbe}
    \end{subfigure}
    \caption{Apprentissage d'un 7 sans bruit en sortie (100 000 parties)}
    \label{fig:Wgan_result_1}
\end{figure}
	La figure \ref{fig:Wgan_result_1_nombres} montre le résultat de la structure précédente. La première conclusion est que l'algorithme peut fonctionner et donner des résultats, ce qui est très encourageant. On observe que les 7 sont biens formés, que la délimitation avec la partie blanche n'est pas nette, et qu'il ont tous sensiblement la même forme. On est donc dans un mode collapse qui se confirme lorsque l'on regarde l'évolution. Un seul 7 est visible à chaque instant, mais sa forme évolue constamment entre plusieurs 7 différents.\\
	La figure \ref{fig:Wgan_result_1_courbe} montre l'évolution du score au cours de ce même apprentissage. On y observe bien la pertinence entre la distance de Wasserstein (le score si les convergence sont bien faites) et la qualité visuelle des chiffres. \\
	La forte augmentation au début correspond à la phase d'initialisation, où l'on ne fait apprendre que le critique afin de calculer la divergence initiale. On observe au cours de l'apprentissage l'amélioration de l'image de façon synchrone avec la chute du score, avec une très rapide progression au début (pour les contours grossier), puis une progression plus limitée (pour les contours fins), et enfin une stagnation visuelle à partir de 25 000 apprentissages.

	L'étape suivante consiste à tenter d'obtenir de la diversité, soit en s’éloignant du mode Collapse, soit en produisant plusieurs chiffres (objectifs en réalité très similaires). 

	Nous avons introduit du bruit en couche de sortie (comme vu au chapitre 2.7) afin d'observer de la diversité. En tentant de produire plusieurs chiffres, nous n'avons eu aucun résultat concluant, seul une tache centrale persiste, que l'on peut peut être interpréter comme une moyenne des chiffres existants. Avec un seul chiffre, on obtient le résultat de la figure \ref{fig:Wgan_result_2_nombres}, on parvient donc à sortir du Collapse, mais on perd en qualité visuelle, le tracé n'étant plus lisse, mais fragmenté.

	Attention, ces derniers résultats sont à prendre avec beaucoup de précaution, ils ont été faits rapidement en fin de projet et n'ont donc pas pu être affinés. Ainsi les conclusions sont que le WGAN fonctionne, que les métriques qu'ils proposent semblent pertinentes vis à vis de ce qui est présenté dans le papier, mais qu'il est soumis à une part importante des problèmes des GANs, une intolérance vis à vis des paramètres approximatifs et le mode Collapse.

\begin{figure}[ht!]
  \centering
  	\begin{subfigure}[b]{.4\linewidth}
    \includegraphics[width=5cm]{fig/Wgan_result_2}
    \caption{Résultat du WGAN pour 50 000 parties, D(x) = 0.58)}
    \label{fig:Wgan_result_2_nombres}
\end{subfigure}
\quad
\begin{subfigure}[b]{.4\linewidth}
  \centering
    \includegraphics[width=5cm]{fig/Wgan_result_2_courbe}
    \caption{Distance de Wasserstein pendant l'apprentissage}
    \label{fig:Wgan_result_2_courbe}
    \end{subfigure}
    \caption{Apprentissage d'un 7 avec bruit en sortie (100 000 parties)}
    \label{fig:Wgan_result_2}
\end{figure}
 \section{Réflexion sur l'approche}




 \subsection{Utilisation de la distance de Wasserstein pour évaluer un générateur}
 \subsection{Problème du caractère Lipschitziens des réseaux de neurones}
 \subsection{Peut-on adapter la logique de cette algorithme à d'autre méthode ?}
% appendices
\appendix

% insérez les annexes ici (\includes)

\bibliographystyle{plain}
\bibliography{reference}


\backmatter

% éventuellement: liste de figures et de tableaux
%\listoffigures
%\listoftables

\end{document}