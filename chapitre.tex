% \begin{figure}[htb]
% \begin{center}
% %optional pour enlever un peu d'espace blanc de votre silhouette
% %\vspace{-.3cm}
%  %\includegraphics[keepaspectratio,width=0.5\textwidth]{fig/nomde}
% % analoog
% %\vspace{-0.6cm}
%  % \caption{ici un logo}
%  % \label{fig:ruglogo}
% %analoog
% %\vspace{-.6cm}
% \end{center}
% \end{figure}

% Utilisation du template
% \cite(reference à citer)
% \ref(figure à référencer)


\chapter{Introduction aux réseaux de neurones et premières applications}

\section{Outils utilisés pour le projet}

Description des outils mis en place pour le projet et de certains choix techniques.

\section{Réseaux de neurones par couches : le perceptron}

\subsection{Structure du réseau}
\subsection{apprentissage par rétropropagation}



\section{Application au problème du XOR}

\paragraph*{}
Lorsque l'on souhaite travailler sur des algorithmes d'apprentissages par ordinateur il est recommandé de les essayer sur des problèmes connus afin d'en vérifier les performances. \\
Le problème du XOR est l'un des plus classiques car il apporte de nombreuses difficultés.\\

L'objectif du XOR est de séparer le plan complexe en quatres cadrants, $(x >0, y > 0)$, $ (x>0, y<0)$, $ (x<0, y>0)$ et $ (x>0, y<0) $. On restreint le plan à $[-1;1]^2$. Les sorties attendues par le réseau de neurones sont 1 pour les points tel que $x*y > 0 $ et -1 pour les points tels que $x*y<0$. \\
Le premier intérêt de ce problème est qu'il est non linéaire, c'est à dire que l'on ne peut pas tracer une droite séparant le plan en 2 qui répond à celui-ci.\\

C'est en se basant sur la résolution du XOR que nous avons construits notre structure de réseau et vérifié la cohérence de notre code. La littérature propose comme réseau le plus simple pour ce problème une couche cachée de 2 neurones, avec 2 entrées ($x$ et $y$) et 1 sortie dans $[-1, 1]$






