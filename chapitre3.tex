%!TEX root = main.tex
% \begin{figure}[htb]
% \begin{center}
% %optional pour enlever un peu d'espace blanc de votre silhouette
% %\vspace{-.3cm}
%  %\includegraphics[keepaspectratio,width=0.5\textwidth]{fig/nomde}
% % analoog
% %\vspace{-0.6cm}
%  % \caption{ici un logo}
%  % \label{fig:ruglogo}
% %analoog
% %\vspace{-.6cm}
% \end{center}
% \end{figure}

% Utilisation du template
% \cite(reference à citer)
% \ref(figure à référencer)

\chapter{Améliorations classiques des Réseaux de neurones appliqués à GAN}


\paragraph{}
Les réseaux de neurones que nous utilisons pour le GAN sont de simples perceptrons. De nombreuses méthodes pour améliorer les résultats et/ou la convergence ont été proposés pour ces types de réseaux. Nous avons étudié en particulier les algorithmes de descente de gradient avec pas adaptatif et les réseaux de neurones à convolution. 

\section{Algorithmes de descente de gradient à pas adaptatif}
\paragraph{}
En utilisant la descente de gradient classique, nous avons constaté que, dans le générateur, presque seule la couche de sortie travaillait. Le GAN ne générait alors pas des images d'assez bonne qualité ni assez diverses. Nous nous sommes donc intéressés à d'autres algorithmes de descente, dans l'espoir qu'ils soient plus efficaces et atteignent plus "en profondeur" les réseaux. 
\paragraph{}
Les algorithmes de descente auxquels nous nous sommes intéressés sont notamment des algorithmes à pas adaptatif. En effet, dans ces algorithmes, le pas change au fur et à mesure de l'apprentissage. Il peut être grand au début, pour aller dans la bonne direction, et petit à la fin pour plus de précision. 

\subsection{Momentum}
Dans une descente de gradient classique, la formule de mise à jour des poids est la suivante. 
\[W_{k+1} = W_k - \eta * \frac{\partial J}{\partial W}\]
On peut également le noter :
\[\Delta W_k = -\eta*\frac{\partial J}{\partial W}\]

Une première méthode qui est compatible avec tous les algorithmes suivants est de rajouter une inertie au gradient, ou momentum. Le but est de limiter les oscillations "inutiles" qui peuvent arriver lors d'une descente de gradient. On a alors : 
\[\Delta W_k  = \mu * W_k - \eta*\frac{\partial J}{\partial W}\]

\subsection{AdaGrad}
AdaGrad (qui signifie Adaptative Gradient) est un algorithme où le pas change en fonction de l'erreur. 
On calcule la somme des carrés des gradients : 
\[g_{k+1} = g_k + (\frac{\partial J}{\partial W})^2\]

La formule de mise à jour des poids est alors : 
\[\Delta W_k = -\frac{\partial J}{\partial W}*\frac{\eta}{\sqrt{g_{k+1}}+\epsilon}\]

$\epsilon$ est une valeur arbitrairement faible pour éviter une division par zéro et pour initialiser l'algorithme. L'inconvénient majeur de AdaGrad est que la quantité $g_k$ ne peut qu'augmenter dans le temps, ce qui implique que le pas devient de plus en plus faible. Si l'apprentissage dure trop longtemps, les poids ne bougeront presque plus à cause du faible pas.

\subsection{RMSProp}
RMSProp est sensiblement identique à Adagrad mais avec une amélioration : au lieu de considérer la somme des carrés des gradients, on considère une pondération de cette somme. Cela permet de donner plus d'importance aux derniers gradients. 
On calcule donc : 
\[g_{k+1} = \gamma*g_k + (1-\gamma)*(\frac{\partial J}{\partial W})^2\]

La formule de mise à jour des poids est donc identique à celle d'Adagrad : 
\[\Delta W_k = -\frac{\partial J}{\partial W}*\frac{\eta}{\sqrt{g_{k+1}}+\epsilon}\]

Dans le calcul de $g_k$, il y a un terme quadratique. On appelle donc $g_k$ \textbf{moment d'ordre 2}. Il existe également un moment d'ordre 1 qui se calcule par :
\[g_{k+1} = \gamma*g_k + (1-\gamma)*\frac{\partial J}{\partial W}\]
Cela donne comme équation de mise à jour des poids : 
\[\Delta W_k = -\frac{\partial J}{\partial W}*\frac{\eta}{\sqrt{g_{t+1}-(m_{t+1})^2}+\epsilon}\]


\subsection{Adadelta}
Cet algorithme est similaire à RMSProp et utilise également une somme mobile pour calculer le moment d'ordre 2 du gradient, $g_k$. Cependant, au lieu d'avoir un $\eta$ fixe, on introduit $x_k$, le moment d'ordre 2 de $\Delta W_k$.
\[g_{k+1} = \gamma*g_k + (1-\gamma)*(\frac{\partial J}{\partial W})^2\]
\[x_{k+1} = \gamma*x_k + (1-\gamma)*(\Delta W_k)^2\]

On obtient donc :
\[\Delta W_k = -\frac{\partial J}{\partial W}*\frac{\sqrt{x_k + \epsilon}}{\sqrt{g_{t+1}-(m_{t+1})^2}+\epsilon}\]
\subsection{Adam}
Adam (pour Adaptive Moment Estimation) adapte le pas en fonction des moments d'ordre 1 et 2 du gradient. Notons $m_k$ le moment d'ordre 1 et $v_k$ le moment d'ordre 2.
On les calcule par :
\[m_{k+1} = \beta_1*m_k + (1-\beta_1)*g_k\]
\[v_{k+1} = \beta_2*v_k + (1-\beta_2)*g_k^2\]

Quand $m_k$ et $v_k$ sont initialisés à 0, ils sont biaisés vers 0.
Pour pallier à cela, on considère $\widehat{m_k}$ et $\widehat{v_k}$ : 
\[\widehat{m_{k}} = \frac{m_k}{1-\beta_1^k}\]
\[\widehat{v_{k}} = \frac{v_k}{1-\beta_2^k}\]

On met à jour les poids avec :
\[\Delta W_k = -\frac{\eta}{\sqrt{\widehat{v_k}}+\epsilon}*\widehat{m_k}\]

\subsection{Comparaison des algorithmes sur MNIST}

	Sur la figure \ref{fig:comp_algos}, on voit que ces algorithmes améliorent le résultat sur MNIST autant en précision qu'en vitesse de convergence par rapport à l'algorithme de descente de gradient classique. Le meilleur algorithme à utiliser est Adam avec $\eta = 0.01$, $\gamma_1=0.9$ et $\gamma_2 = 0.999$ comme paramètres. Cependant, comme l'a étudié le groupe Couleuvre, il peut être avantageux d'utiliser deux algorithmes différents sur le générateur et le discriminateur. En effet, utiliser Adam sur le générateur et RMSProp sur le discriminateur permet de "ralentir" le discriminateur qui a tendance à devenir meilleur que le générateur. 
	
\begin{figure}[ht!]
  \includegraphics[width=\linewidth]{fig/comparaisonAlgos.png}
  \caption{Comparaison des algorithmes de descentes}
  \label{fig:comp_algos}
\end{figure}

\section{Réseaux à convolution: DCGAN}

Les perceptrons ne tiennent pas compte du type de données en entrées. 
Les réseaux à convolution sont eux plus adaptés au traitement d'images et à la reconnaissance de motifs.
Nous avons donc essayer d'implémenter ce type de réseau.

\subsection{Principe}
Comme le nom l'indique, ces réseaux consistent à appliquer différentes convolutions de matrices sur les images afin d'extraire des motifs tels que des contours, des coins, des changements de couleurs, etc.
\\
Une image est dans ce type de réseau est de dimension $p \times n \times n$, avec $p$ le nombre de canaux (aussi appelé channels ou encore feature maps) de l'image.
Le filtre d'une couche est de dimension $q \times p \times m \times m$, avec $p$ le nombre de canaux en entrée et $q$ le nombre de canaux en sortie.
Un canal contient généralement un type d'information telles que les valeurs RVB de l'image, un contour, un motif, etc.
\\
Mathématiquement, on a alors l'équation de propagation suivante :
\[x^{l}_{i,j}(k) = \sum^{p-1}_{r=0}{\sum^{m-1}_{a=0}{\sum^{m-1}_{b=0}{w_{a,b}^{l,r}y^{r}_{(i+a),(j+b)}(k-1)}}},\]
où $l\in [0, q-1]$ et $(i,j)\in [0, n-m+1]^2$.
\\
On obtient alors les équations de rétropropagation suivantes :
\[\frac{\partial E}{\partial w_{a,b}^{l,r}} = \sum^{n-m}_{i=0}{\sum^{n-m}_{j=0}{\frac{\partial E}{\partial x^{l}_{i,j}(k)}y^{r}_{(i+a),(j+b)}(k-1)}}\]
\[\frac{\partial E}{\partial x_{i,j}^{l}(k)} = \frac{\partial E}{\partial y_{i,j}^{l}(k)}\sigma ' (x^{l}_{i,j}(k))\]
\[\frac{\partial E}{\partial y_{i,j}^{l}(k-1)} = \sum^{q-1}_{r=0}\sum^{m-1}_{a=0}{\sum^{m-1}_{b=0}{\frac{\partial E}{\partial x^{r}_{(i-a),(j-b)}(k)}w^{r,l}_{a,b}(k)}} \]

\subsection{Paramètres des convolutions}
Nous avons vu qu'un filtre est défini par sa surface, et son nombre de canaux en sortie, l'image en entrée imposant les canaux en entrée.
Le nombre de canaux en sortie permet de choisir le nombre de motifs que l'on veut que le réseau soit capable d'identifier.
La surface du filtre permet d'influencer sur les motifs reconnus.
On obtient alors un tenseur de poids pour la couche.
Cependant, il existe d'autres paramètres configurables pour les couches à convolutions.
Ainsi les strides et le zéro-padding sont des paramètres souvent utilisés.
\begin{itemize}
	\item Le zéro-padding consiste à ajouter des zéros autour de l'image.
	Cela peut permettre de donner un peu plus d'importance aux bords de l'image, et permet aussi d'être moins restreint sur la dimension des canaux en sortie du filtre.
	Les dimensions de l'image ne diminuent alors pas nécessairement.

	\item Les strides consiste à faire varier le pas de déplacement du filtre sur l'image, qui vaut 1 par défaut.
	Ce pas variable permet d'adapter la quantité d'information à calculer et à conserver.
\end{itemize}
Pour le choix de la fonction d'activation, la fonction ReLu, ou ses variations telles que PReLu ou LeakyReLu, est souvent conseillée. Cependant, il est nécessaire de mettre une couche de BatchNorm derrière une telle fonction, afin d'éviter que la sortie de la couche diverge trop fortement.
\\
De plus, une couche de pooling est généralement rajoutée après chaque couche de convolution.
Ce pooling permet de concentrer l'information pour mieux l'extraire ensuite.
Différents types de pooling existent tels que le max pooling ou la moyenne.

\subsection{Implémentation}
La mise en place des couches de convolutions a demandé un important travail de réorganisation du code malgré les efforts initiaux pour avoir une base souple et modulaire.
Cette nouvelle implémentation s'inspirent de frameworks tels que TensorFlow, et permet d'enchainer n'importe quel type de couche.
\\
Cependant, bien que cette implémentation tourne à la fois pour les perceptrons classiques et les couches à convolutions, nous n'avons pas réussi à obtenir de résultats convaincants sur la base MNIST ou CIFAR10 avec ces nouvelles couches.
Nous avons alors essayé de reproduire certains réseaux de la littérature, afin de comparer les résultats.
Cependant les réseaux à reproduire étaient assez gros, et ne pouvaient pas terminer en temps raisonnable sur notre implémentation des convolutions qui n'était pas extrêmement optimisée et performante.
De plus, ces réseaux utilisaient souvent des techniques d'apprentissage que nous n'avons pas, tel que le dropout.
\\
Ces différents éléments ne nous ont pas permis d'obtenir des résultats satisfaisants.
 

